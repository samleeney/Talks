\documentclass[11pt]{article}
\usepackage{amsmath}
\usepackage{amssymb}
\usepackage{geometry}
\geometry{margin=1in}

\begin{document}

For detection we only care about the \emph{Bayes factor} between a
signal model $M_1$ and a noise-only model $M_0$,
\[
  K = \frac{Z_1}{Z_0},
  \qquad
  Z_m = p(I \mid M_m)
      = \int p(I \mid \theta, M_m)\,p(\theta \mid M_m)\,d\theta.
\]
All elements of the parameter vector $\theta$ are therefore treated as
\emph{nuisance parameters}: we integrate them out to obtain the
evidence $Z_1$ for the signal model.

\bigskip

\section*{Parameter-wise Treatment}

\subsection*{FRB template and amplitude $A$}

\[
I_{ij}
=
\frac{A}{\sqrt{2\pi}\,\sigma_{\mathrm{frb}}'}
\exp\!\left(
  -\frac{\big[t_j - t_0' - K\,\mathrm{DM}'/f_i^2\big]^2}{2\,{\sigma_{\mathrm{frb}}'}^2}
\right)
+ \epsilon_{ij},
\]

\noindent
where $t_0'$, $\mathrm{DM}'$ and $\sigma_{\mathrm{frb}}'$ are the parameters
that are fitted.  The amplitude $A$ enters the model linearly and can
be treated as a nuisance parameter: for fixed track parameters
$(t_0',\mathrm{DM}',\sigma_{\mathrm{frb}}')$ we either optimise $A$
analytically (least squares) or, with a Gaussian prior on $A$, integrate
it out in closed form.

\subsection*{Noise variance $\sigma_{\mathrm{noise}}^2$}

\bigskip

We assume independent Gaussian noise
\[
\epsilon_{ij} \sim \mathcal{N}(0, \sigma_{\mathrm{noise}}^2),
\]
with a conjugate Inverse-Gamma prior on the variance,
\[
\sigma_{\mathrm{noise}}^2 \sim \mathrm{Inv\text{-}Gamma}(\alpha_0, \beta_0).
\]

For a fixed template (i.e.\ fixed $t_0'$, $\mathrm{DM}'$, $\sigma_{\mathrm{frb}}'$)
and amplitude $A$, define the residuals
\[
r_{ij} = I_{ij} - \mu_{ij}, \qquad
\mu_{ij} = \frac{A}{\sqrt{2\pi}\,\sigma_{\mathrm{frb}}'}
\exp\!\left(
  -\frac{\big[t_j - t_0' - K\,\mathrm{DM}'/f_i^2\big]^2}{2\,{\sigma_{\mathrm{frb}}'}^2}
\right),
\]
and the sum of squared residuals
\[
S = \sum_{i,j} r_{ij}^2.
\]

The Gaussian likelihood for $\sigma_{\mathrm{noise}}^2$ is
\[
p(I \mid t_0', \mathrm{DM}', \sigma_{\mathrm{frb}}', A, \sigma_{\mathrm{noise}}^2)
\propto
(\sigma_{\mathrm{noise}}^2)^{-N_{\mathrm{pix}}/2}
\exp\!\left(-\frac{S}{2\sigma_{\mathrm{noise}}^2}\right),
\]
with $N_{\mathrm{pix}}$ the total number of pixels.  Combining this with
the Inverse-Gamma prior and integrating out $\sigma_{\mathrm{noise}}^2$ gives
the marginal likelihood
\[
p(I \mid t_0', \mathrm{DM}', \sigma_{\mathrm{frb}}', A)
\;\propto\;
\big(\beta_0 + \tfrac{1}{2} S\big)^{-\left(\alpha_0 + \tfrac{1}{2}N_{\mathrm{pix}}\right)},
\]
which is a Student--$t$-type dependence on the residuals $r_{ij}$.

\bigskip

\subsection*{Time index $t_0'$}

For detection we do not need a posterior over $t_0'$; we treat it as a
discrete nuisance parameter.  Given a waterfall with $N_t$ time bins,
we place a prior $p(t_0')$ over the corresponding bin centres (often
uniform) and later sum the evidence contributions over all $t_0'$.

\subsection*{Dispersion and width $(\mathrm{DM}',\sigma_{\mathrm{frb}}')$}

After marginalising over $A$ and $\sigma_{\mathrm{noise}}^2$, the
remaining continuous ``track shape'' parameters are
\[
  \phi = (\mathrm{DM}', \sigma_{\mathrm{frb}}').
\]
For each discrete time-bin index $t_0'$ the contribution to the signal
evidence is the integral
\[
  Z_1(t_0') = \iint p(I \mid t_0', \phi)\,p(\phi)\,d\phi,
\]
where $p(\phi)$ encodes the priors on $\mathrm{DM}'$ and
$\sigma_{\mathrm{frb}}'$.  Writing the log-posterior (up to a constant)
as
\[
  \ell(\phi) = \log p(I \mid t_0', \phi) + \log p(\phi),
\]
the Laplace approximation about its maximiser
\[
  \hat{\phi} = \arg\max_{\phi} \,\ell(\phi)
\]
with Hessian
\[
  H = -\nabla^2_{\phi}\,\ell(\phi)\big|_{\phi = \hat{\phi}}
\]
gives
\[
  \log Z_1(t_0')
  \approx
  \ell(\hat{\phi})
  + \frac{d}{2}\log(2\pi)
  - \frac{1}{2}\log|H|,
  \qquad d = 2.
\]
The full signal evidence then follows by summing over the discrete
time bins,
\[
  Z_1 = \sum_{t_0'} p(t_0')\,Z_1(t_0'),
\]
with $p(t_0')$ a prior over the time-bin index (e.g.\ uniform).

\end{document}
