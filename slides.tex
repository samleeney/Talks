\documentclass[aspectratio=169]{beamer}
\usepackage{animate}
\usepackage{tikz}
\usepackage{svg}
\usepackage{graphicx}
\usetheme{metropolis}
\usepackage{tcolorbox}

% Footer template
\setbeamertemplate{footline}{
    \leavevmode%
    \hbox{%
        \begin{beamercolorbox}[wd=.33\paperwidth,ht=2.5ex,dp=1ex,leftskip=3mm]{author in head/foot}%
            \tiny Sam Leeney
        \end{beamercolorbox}%
        \begin{beamercolorbox}[wd=.33\paperwidth,ht=2.5ex,dp=1ex,center]{title in head/foot}%
            \tiny sakl2@cam.ac.uk
        \end{beamercolorbox}%
        \begin{beamercolorbox}[wd=.33\paperwidth,ht=2.5ex,dp=1ex,rightskip=3mm]{date in head/foot}%
            \tiny \href{https://github.com/samleeney}{github.com/samleeney/Talks} \hfill \insertframenumber/\inserttotalframenumber
        \end{beamercolorbox}%
    }%
    \vskip0pt%
}

\begin{document}

\begin{frame}
    \begin{center}
        {\huge Fast, Marginalised Bayesian Transient Searching\par}
        \vspace{0.4cm}

        {\large Sam Leeney\par}
        {\small Handley Lab, University of Cambridge\par}
        \vspace{0.3cm}

        {\footnotesize Collaborators and discussion partners across SKA/CHIME\par}

        \vfill
        \includesvg[width=0.9\textwidth]{images/affiliations.svg}
    \end{center}
    \vfill
\end{frame}

\begin{frame}{\small{Key message}}
    \begin{tcolorbox}[colback=blue!5!white,colframe=blue!75!black,title=Key message]
        Searching directly on raw time–frequency data with a marginalised Bayesian model lets us keep information, model anomalies jointly, and stay physics-aware.
    \end{tcolorbox}
    \vspace{0.35cm}
    \begin{itemize}
        \item Compression loses power; search on raw cubes to avoid throwing signals away.
        \item Fit and flag together: anomaly probability lives inside the likelihood, not after the fact.
        \item Use physics-informed priors (DM, width, arrival time) to steer the search space.
        \item The math parallelises; GPU acceleration makes this practical.
    \end{itemize}
\end{frame}

\section{What are radio transients?}
\begin{frame}{\small{Fast radio bursts}}
    \centering
    \includegraphics[width=0.9\textwidth,height=0.8\textheight,keepaspectratio]{frb_example.png}
\end{frame}

\begin{frame}{\small{Pulsars and repeating bursts}}
    \centering
    \includegraphics[width=0.9\textwidth,height=0.8\textheight,keepaspectratio]{pulsar_example.png}
\end{frame}

\section{What do they look like in data?}
\begin{frame}{\small{Dynamic spectrum before dedispersion}}
    \begin{columns}[T]
        \begin{column}{0.58\textwidth}
            \includegraphics[width=\textwidth]{real_pulsar.png}
        \end{column}
        \begin{column}{0.38\textwidth}
            \begin{itemize}
                \item Typically we see transients like in the previous images.
                \item At the telescope they look like this: dispersed across frequency.
                \item Also mixed with interference.
                \item Typical pipelines compress this data before fitting; we want to operate in this space entirely.
            \end{itemize}
        \end{column}
    \end{columns}
\end{frame}

\section{Standard search algorithms}
\begin{frame}[plain]
    \begin{center}
        \vfill
        {\Large How do standard search algorithms work?}

        \vspace{0.35cm}
        {\small Next: RFI flagging, dedispersion sweeps, and S/N thresholding in practice}
        \vfill
    \end{center}
\end{frame}

\begin{frame}{\small{Simulated FRB observation (raw)}}
    \begin{columns}[T]
        \begin{column}{0.58\textwidth}
            \includegraphics[width=\textwidth]{images/standard_search_raw.png}
        \end{column}
        \begin{column}{0.38\textwidth}
            \textbf{What you see}
            \begin{itemize}
                \item Dispersed sky pulse buried under thermal noise.
                \item Broadband glitch and narrowband line RFI dominate the collapsed S/N.
                \item No dedispersion yet: pulse power is smeared out in time.
            \end{itemize}
        \end{column}
    \end{columns}
\end{frame}

\begin{frame}{\small{Step 1: manual RFI flagging}}
    \begin{columns}[T]
        \begin{column}{0.58\textwidth}
            \includegraphics[width=\textwidth]{images/standard_search_flagged.png}
        \end{column}
        \begin{column}{0.38\textwidth}
            \textbf{Effect on the data}
            \begin{itemize}
                \item Mask out the obvious broadband burst and persistent narrowband line.
                \item Residual dynamic spectrum still contains the dispersed pulse + noise.
                \item Collapsed S/N improves slightly but is still below threshold.
            \end{itemize}
        \end{column}
    \end{columns}
\end{frame}

\begin{frame}{\small{Step 2: dedispersion sweep (low DM)}}
    \centering
    \includegraphics[width=0.85\textwidth,height=0.82\textheight,keepaspectratio]{images/standard_search_dm050.png}
\end{frame}

\begin{frame}{\small{Dedispersion sweep (DM 100)}}
    \centering
    \includegraphics[width=0.85\textwidth,height=0.82\textheight,keepaspectratio]{images/standard_search_dm100.png}
\end{frame}

\begin{frame}{\small{Dedispersion sweep (still low DM)}}
    \centering
    \includegraphics[width=0.85\textwidth,height=0.82\textheight,keepaspectratio]{images/standard_search_dm150.png}
\end{frame}

\begin{frame}{\small{Dedispersion sweep (DM 200)}}
    \centering
    \includegraphics[width=0.85\textwidth,height=0.82\textheight,keepaspectratio]{images/standard_search_dm200.png}
\end{frame}

\begin{frame}{\small{Dedispersion sweep (DM 250)}}
    \centering
    \includegraphics[width=0.85\textwidth,height=0.82\textheight,keepaspectratio]{images/standard_search_dm250.png}
\end{frame}

\begin{frame}{\small{Dedispersion sweep (closer DM)}}
    \centering
    \includegraphics[width=0.85\textwidth,height=0.82\textheight,keepaspectratio]{images/standard_search_dm300.png}
\end{frame}

\begin{frame}{\small{Dedispersion sweep (DM 380)}} 
    \centering
    \includegraphics[width=0.85\textwidth,height=0.82\textheight,keepaspectratio]{images/standard_search_dm380.png}
\end{frame}

\begin{frame}{\small{Dedispersion sweep (DM 440)}} 
    \centering
    \includegraphics[width=0.85\textwidth,height=0.82\textheight,keepaspectratio]{images/standard_search_dm440.png}
\end{frame}

\begin{frame}{\small{Dedispersion sweep (approaching DM)}} 
    \centering
    \includegraphics[width=0.85\textwidth,height=0.82\textheight,keepaspectratio]{images/standard_search_dm520.png}
\end{frame}

\begin{frame}{\small{Dedispersion sweep (over-corrected)}} 
    \centering
    \includegraphics[width=0.85\textwidth,height=0.82\textheight,keepaspectratio]{images/standard_search_dm700.png}
\end{frame}

\section{Bayesian alternative}
\begin{frame}{\small{Idea: keep the raw information}}
    \begin{itemize}
        \item Choose search parameters from physics-informed priors (DM, width, arrival window).
        \item Fit and flag simultaneously so no information is discarded to pre-filtering.
        \item Stay in the native time–frequency cube—no irreversible compression first.
        \item Make it GPU-fast by keeping the marginalisations analytic and parallel.
    \end{itemize}
\end{frame}

\begin{frame}{\small{Step 1: generative model}}
    \[
    I_{ij}
    =
    \frac{A}{\sqrt{2\pi}\,\sigma_{\mathrm{frb}}'}
    \exp\!\left(
      -\frac{\big[t_j - t_0' - K\,\mathrm{DM}'/f_i^2\big]^2}{2\,{\sigma_{\mathrm{frb}}'}^2}
    \right)
    + \epsilon_{ij},
    \]
    \begin{itemize}
        \item Likelihood: $\epsilon_{ij} \sim \mathcal{N}(0, \sigma_{\text{noise}}^2)$; evaluate on the full cube.
        \item Sample with nested sampling to obtain posteriors and evidence for model comparison.
    \end{itemize}
\end{frame}

\begin{frame}{\small{Step 1: simple vs. RFI-contaminated pulse}}
    \begin{columns}[T]
        \begin{column}{0.48\textwidth}
            \centering
            \includegraphics[width=\linewidth,height=0.72\textheight,keepaspectratio]{images/step1_simulated_pulse.png}
            \\[-0.1cm]
            {\scriptsize Clean simulated pulse}
        \end{column}
        \begin{column}{0.48\textwidth}
            \centering
            \includegraphics[width=\linewidth,height=0.72\textheight,keepaspectratio]{images/step1_simulated_pulse_rfi.png}
            \\[-0.1cm]
            {\scriptsize Same pulse with strong RFI (not yet modelled)}
        \end{column}
    \end{columns}
\end{frame}
\section{Bayesian anomaly detection}
\begin{frame}{\small{Bayesian anomaly detection (Bernoulli mask)}}
    \begin{itemize}
        \item Each pixel carries a latent mask $\epsilon_{ij}\!\in\!\{0,1\}$: $\epsilon\!=\!1$ marks an anomaly (RFI), $\epsilon\!=\!0$ is nominal.
        \item Prior on the mask: $P(\epsilon_{ij}) = p^{\epsilon_{ij}} (1-p)^{1-\epsilon_{ij}}$ with $p$ as the anomaly rate.
        \item Full likelihood (before marginalisation):
        \[
        P(\mathbf{I},\epsilon\mid\theta,p) = \prod_{ij} \big[(1-p)L_{ij}(\theta)\big]^{1-\epsilon_{ij}}\!\left(\frac{p}{\Delta}\right)^{\epsilon_{ij}},
        \]
        where $L_{ij}(\theta)$ is the nominal Gaussian likelihood and $\Delta$ is a broad anomaly scale.
    \end{itemize}
\end{frame}

\begin{frame}{\small{Marginalising anomalies without $2^{N}$ masks}}
    \begin{itemize}
        \item Exact marginal in log-space (only expression we keep):
        \[
        \log P(\mathbf{I}\mid\theta,p)=\sum_{ij}\log\!\left((1-p)L_{ij}(\theta)+\frac{p}{\Delta}\right).
        \]
        \item Expanding shows the Occam term that downweights extra mask freedom:
        \[
        \log P=\underbrace{\sum_{ij}\log\!\big((1-p)L_{ij}\big)}_{\text{fit term}}+\underbrace{\sum_{ij}\log\!\left(1+\frac{p/\Delta}{(1-p)L_{ij}}\right)}_{\text{Occam penalty}}.
        \]
        \item Maximising over $\epsilon$ for intuition gives
        \[
        \log P(\mathbf{I}\mid\theta,\epsilon^{\max})=\sum_{ij}\log\!\Big(\max\!\big((1-p)L_{ij},\,p/\Delta\big)\Big),
        \]
        and the marginal log-evidence sits below this by the penalty term above.
        \item Result: a tractable likelihood that jointly fits the pulse and flags RFI probabilistically, yielding evidence $\mathcal{Z}_{\text{signal+anomaly}}$ for detection.
    \end{itemize}
\end{frame}

\section{Real data fits}
\begin{frame}{\small{Run on real data (flagged anomalies)}}
    \begin{columns}[T]
        \begin{column}{0.48\textwidth}
            \includegraphics[width=\textwidth,height=0.8\textheight,keepaspectratio]{real_pulsar.png}
        \end{column}
        \begin{column}{0.48\textwidth}
            \includegraphics[width=\textwidth,height=0.8\textheight,keepaspectratio]{real_pulsar_flagged.png}
        \end{column}
    \end{columns}
\end{frame}

\begin{frame}{\small{Run on real data (fit)}}
    \centering
    \includegraphics[width=0.88\textwidth,height=0.8\textheight,keepaspectratio]{real_pulsar_fitted.png}
\end{frame}

\begin{frame}{\small{Flagging detections}}
    \begin{itemize}
        \item We flag when the Bayes factor exceeds 1:
        \[
        \mathcal{B} = \frac{\mathcal{Z}_{\text{signal+anomaly}}}{\mathcal{Z}_{\text{noise-only}}} > 1.
        \]
        \item Posterior on $p$ shows how much of the cube was treated as interference.
    \end{itemize}
\end{frame}

\begin{frame}{\small{Problem… too slow}}
    \begin{itemize}
        \item Even with JAX/Numpyro-style acceleration, full inference takes minutes per candidate.
        \item At common resolutions, evaluating the Bayes factor can take minutes per ms of data.
        \item We need an analytic marginalisation for the evidence to keep up with survey rates.
    \end{itemize}
\end{frame}

\section{Analytic Bayes factor}
\begin{frame}{\small{Goal: closed-form Bayes factor}}
    \[
    K = \frac{Z_1}{Z_0}, \qquad Z_m = \int p(I\mid\theta, M_m)\,p(\theta\mid M_m)\,d\theta
    \]
    \begin{itemize}
        \item Treat all template parameters as nuisance; integrate them out analytically where possible.
        \item Keep only a small numerical search over $(\mathrm{DM}', \sigma_{\mathrm{frb}}')$ and discrete $t_0'$.
        \item Result: fast, calibrated evidence for detection.
    \end{itemize}
\end{frame}

\begin{frame}{\small{Signal model (from Step 1)}}
    \[
    I_{ij}
    =
    \frac{A}{\sqrt{2\pi}\,\textcolor{red}{\sigma_{\mathrm{frb}}'}}
    \exp\!\left(
      -\frac{\big[t_j - \textcolor{red}{t_0'} - K\,\textcolor{red}{\mathrm{DM}'}/f_i^2\big]^2}{2\,\textcolor{red}{{\sigma_{\mathrm{frb}}'}^2}}
    \right)
    + \epsilon_{ij}, \quad \epsilon_{ij}\sim\mathcal{N}(0,\sigma^2)
    \]
    \begin{itemize}
        \item Parameters to marginalise: $t_0'$ (discrete), amplitude $A$, noise $\sigma$, track shape $(\mathrm{DM}',\sigma_{\mathrm{frb}}')$.
    \end{itemize}
\end{frame}

\begin{frame}{\small{Visualising $t_0'$ (arrival asymptote)}}
    \centering
    \includegraphics[width=0.9\textwidth,height=0.78\textheight,keepaspectratio]{images/t0_visual.png}
\end{frame}

\begin{frame}{\small{Marginalising $t_0'$ (discrete time index)}}
    \[
    I_{ij}
    =
    \frac{A}{\sqrt{2\pi}\,\sigma_{\mathrm{frb}}'}
    \exp\!\left(
      -\frac{\big[t_j - \textcolor{red}{t_0'} - K\,\mathrm{DM}'/f_i^2\big]^2}{2\,{\sigma_{\mathrm{frb}}'}^2}
    \right)
    + \epsilon_{ij}, \quad \epsilon_{ij}\sim\mathcal{N}(0,\sigma^2)
    \]
    \begin{itemize}
        \item Treat $t_0'$ as a discrete prior over time bins (often uniform).
        \item Condition on $t_0'$ and evaluate in parallel across time bins:
        \[
        Z_1(t_0') = \iint p(I\mid t_0',A,\sigma,\mathrm{DM}',\sigma_{\mathrm{frb}}')\,p(A,\sigma,\mathrm{DM}',\sigma_{\mathrm{frb}}')\,dA\,d\sigma\,d\phi
        \]
        with $\phi=(\mathrm{DM}',\sigma_{\mathrm{frb}}')$.
        \item Full evidence: $Z_1 = \sum_{t_0'} p(t_0')\,Z_1(t_0')$; embarrassingly parallel over $t_0'$.
    \end{itemize}
\end{frame}

\begin{frame}{\small{Marginalising amplitude $A$}}
    \[
    I_{ij}
    =
    \frac{\textcolor{red}{A}}{\sqrt{2\pi}\,\sigma_{\mathrm{frb}}'}
    \exp\!\left(
      -\frac{\big[t_j - t_0' - K\,\mathrm{DM}'/f_i^2\big]^2}{2\,{\sigma_{\mathrm{frb}}'}^2}
    \right)
    + \epsilon_{ij}, \quad \epsilon_{ij}\sim\mathcal{N}(0,\sigma^2)
    \]
    \begin{itemize}
        \item Write $\mu_{ij} = A\,\psi_{ij}(t_0',\mathrm{DM}',\sigma_{\mathrm{frb}}')$.
        \item Gaussian prior on $A$ $\Rightarrow$ complete the square and integrate:
        \[
        A\mid \psi,I,\sigma^2 \sim \mathcal{N}(\hat{A},\,\Lambda_A^{-1}), \quad \hat{A} = \frac{\psi^\top I}{\psi^\top\psi+\sigma_A^{-2}\sigma^2}.
        \]
        \item Plugging back yields a marginal likelihood depending only on $\sigma^2$ and $\phi$.
    \end{itemize}
\end{frame}

\begin{frame}{\small{Marginalising noise $\sigma^2$}}
    \[
    I_{ij}
    =
    \frac{A}{\sqrt{2\pi}\,\sigma_{\mathrm{frb}}'}
    \exp\!\left(
      -\frac{\big[t_j - t_0' - K\,\mathrm{DM}'/f_i^2\big]^2}{2\,{\sigma_{\mathrm{frb}}'}^2}
    \right)
    + \epsilon_{ij}, \quad \epsilon_{ij}\sim\mathcal{N}(0,\textcolor{red}{\sigma^2})
    \]
    \begin{itemize}
        \item Prior $\sigma^2 \sim \mathrm{Inv}\text{-}\Gamma(\alpha_0,\beta_0)$, likelihood Gaussian on residuals.
        \item After integrating out $A$, residual sum of squares $S$ gives:
        \[
        p(I\mid t_0',\phi) \propto \big(\beta_0 + \tfrac{1}{2}S\big)^{-(\alpha_0 + N_{\mathrm{pix}}/2)}.
        \]
        \item Noise variance disappears analytically; no sampler needed.
    \end{itemize}
\end{frame}

\begin{frame}{\small{Laplace for $(\mathrm{DM}',\sigma_{\mathrm{frb}}')$}}
    \[
    I_{ij}
    =
    \frac{A}{\sqrt{2\pi}\,\textcolor{red}{\sigma_{\mathrm{frb}}'}}
    \exp\!\left(
      -\frac{\big[t_j - t_0' - K\,\textcolor{red}{\mathrm{DM}'}/f_i^2\big]^2}{2\,{\textcolor{red}{\sigma_{\mathrm{frb}}'}}^2}
    \right)
    + \epsilon_{ij}, \quad \epsilon_{ij}\sim\mathcal{N}(0,\sigma^2)
    \]
    \begin{itemize}
        \item Remaining continuous track parameters $\phi = (\mathrm{DM}', \sigma_{\mathrm{frb}}')$.
        \item Laplace about $\hat{\phi}$ with Hessian $H$:
        \[
        \log Z_1(t_0') \approx \ell(\hat{\phi}) + \frac{d}{2}\log(2\pi) - \tfrac{1}{2}\log|H|,\quad d=2.
        \]
        \item Sum over $t_0'$ and form $K = Z_1/Z_0$ for a fast Bayes factor per candidate.
        \item Cuts evidence time from minutes to ${\sim}$10s of ms per candidate; still need to fold in RFI handling.
    \end{itemize}
\end{frame}

\begin{frame}{\small{Speed-up from analytic evidence}}
    \centering
    \includegraphics[width=0.8\textwidth,height=0.78\textheight,keepaspectratio]{obs_time_per_gpu_second.png}
\end{frame}

\end{document}
